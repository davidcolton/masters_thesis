% Chapter Template

\chapter{Data Extraction, Exploration and Processing} % Main chapter title

\label{chapter4} % Change X to a consecutive number; for referencing this chapter elsewhere, use \ref{ChapterX}

\lhead{Chapter 4. \emph{Data Extraction and Processing}} 

This chapter introduces the dataset used in this research project. The goal of this chapter is to outline the steps undertaken to transform the raw HTML, scraped from the Ask.fm website, into a dataset that is a suitable starting point to undertake a text mining research project. From the initial processing and data analysing steps, through to the final data cleansing steps, the progression from unstructured data into structured text is shown. 

The chapter begins with Section \ref{section:data_extraction} describing how the raw data was extracted from the Ask.fm website as well as the selection criteria used to determine which user accounts would be suitable for scraping. This raw data from Ask.fm, initially held in a single HTML file per user account, is then processed in order to extract the information contained within. An overview of these initial data processing steps are given in Section \ref{section:initial_processing}. This section includes details of the python script used to extract the data, the MySQL database that was used to store the data and also the SPSS process used to partition the data into three separate data blocks.

Section \ref{section:data_classification} gives an overview of the process used in the classification of each question as either bullying or not bullying and also includes detailed descriptions of the various types and categories of cyberbullying to be identified. Also provided are samples of the types of cyberbullying uncovered. Section \ref{section:data_exploration} is an exploration of the primary dataset before Section \ref{section:data_prepatation} gives an in-depth analysis of the processing performed on the data to prepare it for modelling. 
