\section{Cyberbullying: What is it and why should we care}
\label{section:3.1}

According to \citet{nahar_step_2013} bullying has moved out of the school yard and is now causing concern online as cyberbullying. \citet{dadvar_improved_2012} highlight that in this age of digital communications you can have hundreds of virtual friends having never met them face to face. \citet{kontostathis_detecting_2013} suggest that the internet and social media applications are being used, particularly by children and teenagers, as a new way to bully, to cyberbully.
 
We begin this review of the literature with an attempt to understand the phenomenon known as cyberbullying. The psychology of bullying and cyberbullying is well-covered elsewhere and is not within the scope of this paper. However, as the principal area targeted by this research is the automatic detection of cyberbullying, it is important first to understand how this new form of bullying is enacted on its victims and then also consider the impact of it on the victims.

The United States Department of Health and Human Services (DOH) \cite{DOH} highlights that bullying exists where there is either a perceived or actual imbalance of power and there is repeated aggressive behaviour. These behaviours could be verbal, for example teasing or name calling, social, including spreading rumours and exclusionary acts or threatened or actual physical violence. \citet{xu_fast_2012} concur with these descriptions of face to face encounters as being what \citet{dadvar_improved_2012} refer to as traditional bullying. Citing the work of others \cite{archer_integrated_2005} \cite{little_disentangling_2003} \cite{nylund_subtypes_2007} bullying is described as taking many forms but the most common are physical or direct aggression, indirect aggressions like name calling and relational or social aggressions such as exclusion. \citet{xu_fast_2012} also highlights that this bullying can be more than an isolated single event and the intended victim can be repeatedly subjected to the abusive behaviour over time.

The DOH describes cyberbullying as bullying using communication tools such as instant messaging, chat sites and social networks using smart phones, computers or tablets. The DOH also highlights that cyberbullying can't be easily turned off and can happen twenty four hours a day seven days a week. To further exasperate the situation the bullying messages or texts can rapidly spread to a large on-line community. Also, the bully can be anonymous and difficult to trace and completely purging the internet of the offending text or image is next to impossible. The persistent nature of the world wide web and functionality provided by social media sites means that these abusive posts can quickly spread amongst a group and can subsequently be accessed again and again by both the victim and the perpetrator \cite{dadvar_improved_2012} \cite{nahar_effective_2013} \cite{dadvar_towards_2012}. The implications of this is that the offending text or image can continuously reappear causing distress to the victim again and again. The effect of the cyberbullying can be traumatic on the victim leading to trouble sleeping, withdrawing from society, stress and more troubling mental health problems like anxiety, depression and suicidal thoughts. 

The DOH is not alone in their views, there is much support for their views and opinions. For example \citet{dadvar_improving_2013} and \citet{nahar_effective_2013} are both of the opinion that cyberbullying is an electronic act, aggressive in nature, against a victim who is typically unable to defend themselves. \citet{dinakar_modeling_2011} and \citet{rybnicek_facebook_2013} describes these electronic acts as images or text messages posted to social media sites with the sole intent of hurting or embarrassing the victim through these repeated offensive postings. 

It is also important to consider the tone and content of a cyberbullying incident and the various forms that it may take. Apart from the threat of physical violence, or the wishing harm on a person,  \citet{dinakar_common_2012} describe three main categorises of cyberbullying as sexual, racial and direct attacks against a person. Sexist attacks are typically against women or sexual minorities such as gay, lesbian, bi-sexual and transsexual individuals and groups. Racial or cultural attacks are typically against a cultural minority and its traditions. Direct personal cyberbullying attacks a persons intelligence or physical appearance for example their weight, height, appearance or IQ. 

\citet{chen_detecting_2012} describes cyberbullying as communications that disparage an individual or group on the basis of their nationality, ethnicity, colour or race, their gender, sexual orientation or religion. \citet{xu_learning_2012} also includes socio-economic status as a cyberbullying category under which the victim can be targeted. \citet{willard_cyberbullying_2006} \cite{willard_cyberbullying_2007} describes the many forms a cyberbullying attack can take as harassment, flaming, outing, exclusion, flooding, cyberstalking, impersonation or masquerading, trolling and denigration of their victims by the bullies. \citet{ptaszynski_michal_machine_2010} bring to our attention and initiative from the Ministry of Education, Culture, Sports, Science and Technology (MEXT) in Japan. Following several tragic suicides attributed to cyberbullying MEXT developed a manual to assist school teachers to identify cases of cyberbullying \cite{ministry_of_education_culture_sports_science_and_technology_netto_2008}. MEXT divided cyberbullying into two separate categories, cyberbullying posts appearing on blogs, forums and private profile sites and cyberbullying emails. These posts and emails can contain libellous or slanderous content, the unauthorised disclosure of sensitive personal data or posts intended to humiliate the victim.

The affects of cyberbullying on its victims can  sometimes have tragic consequences as previously highlighted. News headlines like ``Third suicide in weeks linked to cyberbullying'' \cite{Cionnaith:2102}, ``Cyberbullies claimed lives of Five teens'' \cite{Riegel:2013} and ``Hanna Smith suicide fuels calls for action on Ask.fm cyberbullying'' \cite{laura_smith-spark_hanna_2013} are now, unfortunately, becoming an all too depressingly frequent occurrence. However, behind these eye-catching headlines the daily torment and abuse suffered by the victims of cyberbullying are taking a dire psychological and emotional toll. The negative affects of posts that contain cyberbullying of a personal or sensitive nature can be internalised by children and young adults leading to significant and emotional psychological suffering says \citet{dinakar_modeling_2011}. Apart from suicidal thoughts \citet{xu_fast_2012} say the affects of cyberbullying can include loneliness, anxiety, low self-worth and signs of depression. Other signs described in \citet{xu_learning_2012} are intra-personal problems, school absence or violence and physical complaints.  

In their report ``Cyberbullying Among 9-16 Year Olds in Ireland'' \citet{oneill_cyberbullying_2013} provide an overview of bullying and cyberbullying statistics in Ireland and Europe. The main findings of their report are that nearly one in four (23\%) of all 9-16 year old children had experienced some bullying whether it was on-line or off-line. Although at 4\% the overall number of Irish teens that experienced on-line bullying was lower than the European average of 6\% it was noted that the more ``experienced internet users'', particularly users of internet social media sites, reported higher levels of bullying. The most commonly reported form of cyberbullying was being the target of hurtful or nasty messages. \citet{dadvar_towards_2012}, however, warns that focusing solely on the text of a message may not provide sufficient evidence to conclude that the text is a cyberbullying post. The use of profanities, and what others would consider inappropriate and offensive language, may just be what the protagonists involved consider good banter.

Other types of unacceptable on-line activities that also target adolescents and teenagers include cyberstalking and grooming. \citet{aggarwal_anti-cyberstalking:_2005} define stalking as the unwanted and repeated observation, harassment, intimidation and intrusion of privacy of a person by a predator who wishes to acquire private information about their victim. They continue that the anonymity and wealth of new stalking opportunities offered by the internet has given the stalkers a feeling of relative safety as the technical ability and internet competency of the stalker typically exceeds that of the victim. \citet{rybnicek_facebook_2013} say that the grooming of a child is typically performed by adults who approach children in order to have sexually explicit conversations, to exchange nude photographs or videos or to meet up in person resulting in the sexual abuse of the child. \citet{elzinga_analyzing_2012} say that there are seven stages of grooming. The first two stages are sweet greetings, where pet names like \textit{``sweety''} are used to create an atmosphere of intimacy, and compliments, which is used to strengthen the intimacy sphere by complimenting the targets looks for example. Stages three and four are called intimate parts and sexual handling where the intimate parts or genitalia of the body are introduced to the conversation by using their popular names, for example \textit{``boobs''} or \textit{``willy''}, before the chat becomes more explicit. In stage five the groomer will use photographs and videos to show his private parts and attempt to get the intended victim to do likewise before in stages six and seven the location, where, and the date, when, the groomer can meet the victim is discussed

So how then can modern machine learning and computing help in the detection of cyberbullying? Some of the approaches used in the detection of cyberbullying are further explored next.